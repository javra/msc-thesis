Summarizing the previous chapters, my work consists of translating the algebraic
structures of double groupoids and crossed module to set based structures in
homotopy type theory.
I transferred the principal example of a fundamental double groupoid of a triple
of spaces from the topological setting to its equivalent in the world of higher
types by defining the fundamental double groupoid of a presented 2-type.
I formalized the structures of double categories, double groupoids and crossed
modules in the new theorem proving language Lean and mechanized the essential
parts of the proof that double groupoids with thin structures and crossed modules
form equivalent categories.
I furthermore made the formalized structures applicable to 2-truncated higher types by
instantiating the fundamental double groupoid of such types.

I hereby made it possible to analyze presented 2-types in purely set-based algebraic
structures.
This opens up the analysis of second homotopy groups or second homotopy groupoids
of 2-types and the characteristics of these groups and groupoids to the use of
formalized group theoretical and category theoretical knowledge.
This could lead to direct computation of several homotopy invariants.
Being one of the first greater formalization projects in Lean, the problems encountered
during the process of writing the formal definitions and proofs led to improvements
in the performance and usability of Lean.
With respect to their compilation time, my theory files also serve as a benchmark
for the elaboration and type checking algorithms used in Lean.

What are the main insights and experiences gained from this work?
Many of the difficulties in writing the formalization are certainly due to the
early development stage of the system used.
During my work, Lean's features for structure definition, type class inference,
tactics, as well as troubleshooting and output were vastly enhanced and improved.
Also, the time Lean needs to elaborate and type check the theory files have decreased
drastically since the start of the project.
The library of category theory and group theory that were developed alongside the
actual formalization project and are still work in progress.
Another big hurdle was the management of transport terms in my proofs.
Even actually trivial proofs involving equalities of two-cells in double
categories and double groupoids turned out to be long and tedious due to the need
of moving transport terms from the inside of an operator or a function to its
outside.
This is, of course, the price one has to pay for the heavy use of dependently
typed two-cells in double categories.
After gaining experience on what auxiliary lemmas were needed and how to use them,
this burden of moving transport terms was reduced to a mere strain on the theorem
prover and a part of the theories that made the files longer and less readable.
Finally, some parts of the proofs which were not stated explicitly but instead
left out as ``trivial'' in my main reference~\cite{nat} turned out to be more
sophisticated than initially anticipated.

There are several points where I could have made a different decision that would
have led to different results in the complexity and the character of my formalization.
As mentioned in Chapter~\ref{chapter:types}, the decision on whether to formalize
the higher cells of a double category as dependent types or as flat types with
face operators is a difficult one.
Deciding for flat types would have prevented the need for many auxiliary lemmas
involved in the effort to control transport terms in equalities of two-cells.
It is hard to judge whether a flat typed approach would have led to longer proofs
and less readable definitions or if it made the formalization cleaner and shorter.
Another way of preventing the need of said auxiliary lemmas and of applying lemmas
to move transport terms to different sides of equations would be using what Daniel
Licata calls ``pathovers'' and ``squareovers'' in his paper~\cite{licatacubical}
describing a strategy for the proof that, as a higher inductive type, the torus
is equivalent to the product of two circles.
These encapsulate the type of equalities like those of the form $p_*(x) = y$ as objects,
formalized as an inductive type. %TODO more details?
Another solution would be to switch to a different logic that allow postulating
judgmental equalities, e.g. the cubical identities (\ref{eq:corner-ident}).
One example for such a logical framework might by Vladimir Voevodsky's
Homotopy Type System (HTS)~\cite{hts}.
This system might make it possible to generalize my formalization to the case
of cubical $\omega$-groupoids and crossed complexes -- something which, with my
current approach -- is not possible since there is no uniform way to describe the
dependent types of $n$-cells for all $n \in \mathbb{N}$.

This leads to the question what could be a possible way to continue and extend
my project.
The most obvious use of the formalization would be a 2-dimensional Seifert-van
Kampen theorem for presented 2-types.
In its most common form, such a theorem would state that the category
theoretical pushout of the fundamental double groupoid of two presented 2-types
is isomorphic to the fundamental double groupoid of the pushout of those presented
2-types in the form of a higher inductive type.
Then, one could search for ways to find ``reasonable'' presentations for 2-types,
homotopy surjective ones come to mind, either manually or automatically at the
time of the definition of a higher inductive type.
Such an automation could then be part of the definitional package of an interactive
theorem prover that provides these higher inductive types as a primitive.
As mentioned in the above paragraph, the most important generalization of my work
would consist of replacing the ``double'' in ``double groupoid'' by ``$n$-fold''
for an arbitrary $n \in \mathbb{N}$ or, as an ultimate goal, by the case of
$\omega$-groupoids that contain higher cells for every dimension.
Finally, one could ask if there are any applications of Ronald Brown's attempts
to ``compute'' crossed modules induced by subgroups \cite{brown1996computing}
to computable homotopy characteristics of higher types.




