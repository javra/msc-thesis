This chapter shall serve to provide the reader with the necessary basic knowledge
about homotopy type theory.
Most of this knowledge was gathered and written up during the ``special year 
on univalent foundations'' which took place in the years 2012 and 2013
at the Institute for Advanced Study in Princeton.
It resulted in the collaborative effort to write and publish a first \emph{book}
on homotopy type theory~\cite{hottbook} which is still being improved and open
for suggestions at GitHub.~\footnote{\url{https://github.com/HoTT/book}}
My description of homotopy type theory will stick to the notation and terminology
used in this book.

Furthermore, I will not make the effort to distinguish between what elements of
the theory were there in earlier approaches to intensional type theory,
most prominently the one of Per Martin-L\"of~\cite{martin-lof1}, as this would
defy the purpose of a concise introduction to the current ``state of the art''.
%TODO this sounds wrong and pretentious?

\section{Some Basic Non-Dependent Type Theory}

A type theoretical foundation of mathematics uses \textbf{Types} wherever,
in an approach based on set theory and predicate logic, sets and propositions
are used.
Homotopy type theory adds to this logical interpretation and set interpretation
the point of view of a topological space or its homotopy type.
\textbf{Objects} (or \textbf{instances}) of a type thus correspond to elements of
a set, to proofs of a proposition, as well as to points in a space.

The judgment that that some object $a$ is an instance of a type $A$ will be written
as $a : A$.
Opposed to set theory it is always a priori determined, what type some constructed
object will be an instance of, and this type is, up to definitional equality of
types, fixed.
If an object or a type can be written in two different ways, we will express the
fact that two expressions coincide using ``$\equiv$'' (since ``$=$'' will later
denote propositional equality).
Likewise, ``$:\equiv$'' is the notation for abbreviating the the right hand side
by the expression on the left hand side.
It is important to note that it is decidable to check whether $a \equiv b$ holds
for two given terms $a$ and $b$.

Types in homotopy type theory are organized in \textbf{universes}.
For every $i \in \mathbb{N}$ we assume to have a universe $\UU_i$ which is itself,
as an object, contained in the universe $\UU_{i+1}$.
In this way, all types, including the universes, can be seen as objects in some
greater type.
Often, it is assumed that universes are \emph{cumulative} in the sense that if
$A : \UU_i$, then $A : \UU_{i+1}$ (and thus, $A : \UU_j$ for every $j \geq i$).
Since this entails some computational difficulties for theorem provers, the
language Lean will not incorporate universe cumulativity.
A replacement for the cumulativity is an inductively defined lifting function
$\UU_i \to \UU_{i+1}$.
In the following, I will most of the time leave the index of a universe implicit
and just denote it by $\UU$. This means to say that these definitions are applicable
to all (combinations of) universe indices.
The reason for the need of multiple universes is that a system that simply assumed
that $\UU : \UU$ would be inconsistent.

We will now take a look at some of the non-dependent type formers, some of which
will later be extended  to dependent ones.
We will introduce these type formers by giving semi-formal rules for the formation
of the type, the introduction of the type's instances and for their elimination.

The most basic type is the type $A \to B$ of \textbf{non-dependent} functions between two
types $A, B : \UU$.
The inference rules that come with it are exactly those known from types $\lambda$-calculus:
\begin{equation}
\begin{gathered}
\inferrule*[left=$\to$-Form]{A,B : \UU}{A \to B : \UU} \quad
\inferrule*[left=$\to$-Intro]{a : A \vdash \Phi[a/x] : B}{(\lambda x. \Phi) : A \to B} \\
\inferrule*[left=$\to$-Elim]{f : A \to B \\ a : A}{f(a) : B}
\end{gathered}
\end{equation}
Here, $\Phi$ is a term that may have $x$ as a free variable.
$\Phi[a/x]$ denotes the replacement by every appearance of $x$ by $a$.
Of course, we have the rules of $\eta$-conversion and $\beta$-reduction:
\begin{equation}
\inferrule*[left=$\eta$]{f : A \to B}{(\lambda x. f(x)) \equiv f} \quad
\inferrule*[left=$\beta$]{(\lambda x. \Phi) : A \to B \\ a : A}
	{(\lambda x. \Phi)(a) \equiv \Phi[a/x]}
\end{equation}
These definitional will from now on be used ``silently'' to replace subterms.
In the logic interpretation of HoTT, function types model implication of propositions
while in the set and topology interpretation they represent (arbitrary resp.
continuous) maps.

\section{Functions and Pi-Types}

\section{Sigma-Types}

\section{Equality}

\section{Truncated Types}

\section{Equivalences and univalence}
