This chapter shall serve to provide the reader with the necessary basic knowledge
about homotopy type theory.
Most of this knowledge was gathered and written up during the ``special year 
on univalent foundations'' which took place in the years 2012 and 2013
at the Institute for Advanced Study in Princeton.
It resulted in the collaborative effort to write and publish a first \emph{book}
on homotopy type theory~\cite{hottbook} which is still being improved and open
for suggestions at GitHub~\footnote{\url{https://github.com/HoTT/book}}.
My description of homotopy type theory will stick to the notation and terminology
used in this book.

Furthermore, I will not make any distinction between elements of homotopy type
theory that were present in earlier approaches to intensional type theory,
most prominently the one of Per Martin-Löf~\cite{martin-lof1} as this
defies the purpose of a concise introduction to the current state of the art.
%TODO this sounds wrong and pretentious
%TODO introduce natural number, unit type, identities, truncation


\section{Functions and Pi-Types}

\section{Sigma-Types}

\section{Equality}

\section{Truncated Types}

\section{Equivalences and univalence}
