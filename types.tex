In this this chapter, I will describe how the structures introduced in the previous
chapter can be translated to homotopy type theory.
Besides formulating the concepts using dependent types this involves caring about
the effects of univalence and proof relevance on these definitions.
What does it mean for two instances of a structure to be propositionally equal?
What truncation levels should be imposed on the parameters of said structure
such that algebraic structures bear no unwanted information in their iterated
equality types?

We start off with defining the notion of a categories, and then continue to
translate the definitions from the previous chapter appropriately.
Finally we will show how to apply the definitions to 2-types and define the
fundamental double groupoid and fundamental crossed module of a presented type.

The standard references for the implementation of categories I will use in this
chapter are the respective chapter of the HoTT-Book~\cite{hottbook} as well as
the paper about ``Univalent Categories and the Rezk Completion'' by Benedikt
Ahrens, Chris Kapulkin and Mike Shulman~\cite{rezk-completion}.
While most of the time I will stick to the (consistent) notation and terminology
of both of these, I will deviate sometimes to bring the presentation more
in line with the actual Lean implementation presented in the next chapter.

\section{Categories in HoTT}
% Here goes stuff about univalent vs non-univalent cats
% maybe rezk-completion

\begin{defn}[Precategory] \label{def:hott-precat}
Let $A : \UU$ be a type (the \textbf{object type} or \textbf{carrier}). A
\textbf{precategory} $C$ on $A$ is constructed by giving the following data:
\begin{itemize}
\item For each $a, b : A$ a type of morphisms $\hom_C(a, b) : U$ for which we furthermore
require that $\prod_{(a,b:A)} \isSet(\hom_C(a, b))$.
\item The composition of morphisms
\begin{equation*}
\comp_C : \prod_{a,b,c:A} \hom_C(b, c) \to \hom_C(a, b) \to \hom_C(a, c) \text{.}
\end{equation*}
We will most of the leave the first three arguments implicit and just write
$g \circ_C f$ or $gf$ for $\comp_C(a,b,c,g,f)$.
\item An identity operator $\id_C : \prod_{(a:A)} \hom_C(a,a)$.
\item A witness ensuring associativity for all morphisms:
\begin{equation*}
\prod_{a,b,c,d:A} ~ \prod_{h:\hom_C(c,d)} ~ \prod_{g:\hom_C(b,c)} ~ \prod_{f:\hom_C(a,b)}
h \circ_C (g \circ_C f) = (h \circ_C g) \circ_C f
\end{equation*}
\item Witnesses that the identiy morphisms are neutral with respect to composition
from the left and from the right:
\begin{equation*}
\prod_{a, b : A} ~ \prod_{f : \hom_C(a, b)}
\left(\id_C(b) \circ_C f = f\right) \times \left(f \circ_C \id(a) = f\right)
\end{equation*}
\end{itemize}
\end{defn}

As with all the definitions given in this semi-informal style, it is, from a
theoretical standpoint, equivalent whether to see them as a description of an
iterated $\Sigma$-Type or as the only constructor of an inducitve type.
We will later~\ref{rmk:inductives-not-sigmas} see that in formalization practice
it is favorable to choose to introduce them as new inductive types instead of
$\Sigma$-Types.

We also observe that, since equalities in sets are mere propositions, we have the
following lemma:
\begin{lemma}[Equality of precategories] \label{thm:hott-eq-precat}
Let $C$ and $D$ Precategories on $A$ with $\hom_D :\equiv \hom_C$, 
$\comp_C = \comp_D$ and $\id_C = \id_D$.
Then, $C = D$. \hfill $\qed$
\end{lemma}

This also justifies that we do not require further coherence conditions on
associativity (the ``pentagon coherence'') and identity laws. %TODO add further explanation?

\begin{defn} [Functors] \label{def:hott-precat-functor}
Let $C_A$ and $C_B$ be precategories on types $A$ and $B$.
A \textbf{functor} $F$ between $C_A$ and $C_B$ is constructed by giving the following:
\begin{itemize}
\item Its definition on objects as a instance of $F_{\obj}: A \to B$.
\item Its definition on morphisms
\begin{equation*}
F_{\hom} : \prod_{a,b:A} ~ \prod_{f:\hom_{C_A}(a,b)} \hom_{C_B}(F_{\obj}(a),F_{\obj}(b))\text{.}
\end{equation*}
Again we will often leave out the first two arguments for $F_{\hom}$ and moreover
abbreviating $F_{\hom}$ and $F_{\obj}$ to $F$ whenever the distinction is clear.
\item A proof in $\prod_{(a:A)} F(\id_{C_A}(a)) = \id_{C_B}(F(a))$ that the identies
are preserved and
\item a proof the respects the composition in the respective categories, as an instance
of
\begin{equation*}
\prod_{a,b,c:A} ~ \prod_{g:\hom_{C_A}(b,c)} ~ \prod_{f:\hom_{C_B}(a,b)}
F(g \circ_{C_A} f) = F(g) \circ_{C_B} F(f)
\end{equation*}
\end{itemize}
\end{defn}

Again, the last two ingredients turn out to be mere propositions, as they are
of $\Pi$-types over equalities in sets.
This leads us to the observation that to prove the equality of two functors, it
suffices to check it on their definitions on objects and morphisms:

\begin{lemma}[Equality of functors]
Let $A, B : \UU$ and let $C$, $D$ be categories on $A$ and $B$, respectively.
Let $F$ and $G$ be two functors from $C$ to $D$. If we have
\begin{align*}
p &: \prod_{a:A} F(a) = G(a) \text{ and } \\
q &: \prod_{a,b:A} ~ \prod_{f : \hom_C(a,b)} p(b)_*\left(p(a)_*(F(f))\right) = G(f) \text{,}
\end{align*}
then $F = G$. \hfill $\qed$
\end{lemma}

With this definition of precategories and functors, a lot of structures can be
instantiated as such.
For example, the 1-types of a universe $\UU_i$ give us a precategory with morphisms
between $A, B : \UU_i$ being $A = B$, composition being concatenation of equalities
and identity being reflexivity.
But often we will only have to deal with precategories whose carrier is a set".

\begin{defn}[Strict precategory] \label{def:hott-strict-precat}
A precategory with a set as carrier is called \textbf{strict}.
\end{defn}

One primary use for strict precategories is the following:
If we wanted to build a precategory of precategories we encounter the problem that
functors between two given precategories don't generally form a set!
Restricting ourselves to strict precategories solves this problem:

\begin{lemma}
Let $C$ be a precategory and $D$ be a strict precategory.
Then, the type of functors between $C$ and $D$ forms a set.
\end{lemma}

\begin{proof}
For the type of functors to be a set, all parameters should be sets.
Since the definition on morphisms is a set by definition and,
as we already observed, the identity witnesses are mere propositions,
the only critical parameter is the object function.
But turning the codomain of this function into a set obviously solves the
problem.
\end{proof}

\begin{corollary}
For each pair of universes $(\UU_i, \UU_j)$ there is a precategory of strict
precategories with carrier in $\UU_i$ and morphism types in $\UU_j$. Morphisms of
this category are functors, with composition of functors and identity functors
being defined as obvious. \hfill $\qed$
\end{corollary}

Another precategory we can consider is the precategory of sets:

\begin{lemma}
Let $\UU_i$ be a universe. Then, the type of sets in $\UU_i$ forms a precategory
with morphisms being arbitrary functions.
\end{lemma}

But the truncation level of the carrier is not the only thing that could be bothersome
when dealing with precategories:
If we look at the previous two examples of precategories where each object itself
are types, we can conclude that if two objects are isomorphic in the algebraic sense
they are equivalent types and thus, by the univalence axiom, equal.
We want to transfer this as requirement to all categories, extending the general
idea of univalence which says that isomorphic objects should be treated as equals.
To make this definition of a \emph{category} as smooth as possible, some helping
will be necessary.

\begin{defn}[Isomorphisms]
Let $C$ be a precategory on a type $A : \UU$. A morphism $f : \hom_C(a, b)$ is
called an \textbf{isomorphism} if there is a morphism $g : \hom_C(b,a)$ that
is both a left and a right inverse to $f$.
Define furthermore $a \cong b$ for $a, b : A$ as the type of all isomorphisms in
$\hom_C(a, b)$.
Formally:
\begin{align*}
\isIso_C(f) &:\equiv \sum_{g:\hom_C(b,a)}
	\left(g \circ_C f = \id_C(a)\right) \times \left(f \circ_C g = \id_C(b)\right) \\
a \cong b &:\equiv \sum_{f:\hom_C(b,a)} \isIso_C(f) \text{.}
\end{align*}
\end{defn}

\begin{lemma}
For each $a, b : A$, $a \cong b$ is a set and for each $f : \hom_C(a, b)$,
$\isIso(f)$ is a mere proposition. \hfill $\qed$
\end{lemma}

Equal objects of a category are always isomorphic: %TODO delete this?

\begin{lemma}
If for two objects $a, b : A$ of a precategory $C$ we have $p : a = b$, then there
exists
\begin{equation*}
\idtoiso_{a,b}(p) : a \cong b \text{,}
\end{equation*}
such that $\idtoiso_{a,a}(\refl_a) = \left(\id_C(a),\ldots\right)$. %TODO be sloppier?
\end{lemma}

\begin{proof}
Induction on $p$ lets us assume that $p \equiv \refl_a$. But since
$\id_C(a) \circ_C \id_C(a) = \id_C(a)$ by either of the cancellation laws for
identities, we get an instance of $a \cong a$.
\end{proof}

\begin{defn}[Category]
A \textbf{category} $C$ on $A$ is defined to be a precategory on $A$ which is
\emph{univalent}: It's accompanied
by a proof that for all $a, b : A$, $\idtoiso_{a,b}$ is an equivalence.
\end{defn}

The essential use of this extension is, of course, the inverse that $\idtoiso$
is assumed to have: Statements about isomorphic objects become statements about
equal objects and can thus be proven using induction. There are some important
examples for univalent categories:

\begin{lemma}
Precategories on $\UU$ and 	precategories on subtypes of $\UU$ (i.e. on
$\sum_{(A : \UU)} P(A)$ where $P : \UU \to \UU$ and $\prod_{(A : \UU)} \isProp(P(A))$)
with function types as morphisms are univalent. \hfill $\qed$
\end{lemma}

\begin{lemma}
The precategory of strict precategories is univalent.
\end{lemma}

%TODO prove this

\begin{lemma} \label{thm:cat-1-type}
If there is a univalent category on a type $A : \UU$, then $A$ is a 1-type.
\end{lemma}

\begin{proof}
For all $a, b : A$, the fact that $a \cong b$ is a set implies that, since
truncation levels are preserved by equivalences, also $a = b$ is a set.
But by definition, this proves that $A$ is 1-truncated.
\end{proof}

A last definition is the one of a groupoid.
Here, the univalent case is rather uninteresting since the structure of the groupoid
is completely determined by the carrier. %TODO is this actually true?

\begin{defn}[(Pre-)Groupoids]
A \textbf{(pre-)groupoid} is a (pre-)category $C$ on $A : \UU$ together with
\begin{equation*}
\alliso_C : \prod_{a,b:A} ~ \prod_{f:\hom_C(a,b)} \isIso(f) \text{.}
\end{equation*}
\end{defn}

\section{Double groupoids in HoTT}

As we saw when introducing (pre-)categories in homotopy type theory, we use
dependent types to model the type of morphisms. Instead, we could have said that
a category is defined on a two types $A, H \in \UU$ where $A$ represents the objects
and $H$ is \emph{one} type of morphisms.
This would have required us to give functions $\partial^-, \partial^+ : H \to A$
specifying domain and codomain of morphisms and the composition to be of the type
\begin{equation*}
\prod_{g,f:H} \left(\partial^+(f) = \partial^-(g) \to 
	\sum_{h:H} \left(\partial^-(h)=\partial^-(f)\right)
		\times \left(\partial^+(h)=\partial^+(g) \right) \right) \text{.}
\end{equation*}
The difference of this approach from the one we chose is that morphisms are only
composable if their respective codomain and domain are \emph{definitionally}
the same. Both approaches are advantageous in some situations:
Using dependent types will force us more often to use transports to achieve
definitional equality on codomain and domain, using $\partial^-$ and $\partial^+$
requires an identity proof for each composition we want to construct.

For the implementation of double categories and double groupoids I decided to
adopt the dependently typed concept that's commonly also used to formalize categories.
Also note that in the following definition we will \emph{not} require the
1-skeleton of a double category to be univalent since in our intended main
use of the structures the object type will not be 1-truncated and thus the
1-skeleton, as seen in lemma~\ref{thm:cat-1-type}, not a univalent category.

\begin{defn}[Double category] \label{def:dbl-cat-hott}
A \textbf{double category} $D$ is constructed by giving the following components:
\begin{itemize}
\item A type $D_0 : \UU$ of objects.
\item A precategory $C$ on $D_0$. To be consistent with the notation in the last
chapter, we will write $D_1(a, b) : \UU$ for $\hom_C(a, b)$.
\item A dependent type of two-cells:
\begin{equation*}
D_2 : \prod_{a,b,c,d:D_0} ~ \prod_{f:D_1(a,b)} ~ \prod_{g:D_1(c,d)}
	~ \prod_{h:D_1(a,c)} ~ \prod_{i:D_1(b,d)} \UU
\end{equation*}
We will always leave the first four parameters implicit and write $D_2(f,g,h,i)$
for the type two-cell with $f$ as its upper face, $g$ as its bottom face,
$h$ as its left face, and $i$ as its right face.
\item The vertical composition operation: For all $a, b, c_1, d_1, c_2, d_2 : D_0$
and $f_1 : D_1(a, b)$, $g_1 : D_1(c_1, d_1)$, $h_1 : D_1(a, c_1)$, $i_1 : D_1(b, d_1)$,
$g_2 : D_1(c_2, d_2)$, $h_2 : D_1(c_1, c_2)$, and $i_2 : D_1(d_1, d_2)$ the composition
of two cells
\begin{equation*}
D_2(g_1, g_2, h_2, i_2) \to D_2(f, g_1, h_1, i_1) 
	\to D_2(f_1, g_2, h_2 \circ h_1, i_2 \circ i_1) \text{.}
\end{equation*}
As this is pretty verbose, we will, from now on, refrain from writing out parameters
that only serve to can easily be inferred from the rest of the term. %TODO express differently

We will denote the vertical composition of $v : D_2(g_1, g_2, h_2, i_2)$ and
$u: D_2(f, g_1, h_1, i_1)$ with $v \circ_1 u$ leaving all other information implicit.
\item The vertical identity $\id_1 : \prod_{(a,b:D_0)} \prod_{(f : D_1(a,b))}
D_2(f,f,\id_C(a),\id_C(b))$.
\item For all $w : D_2(g_2,g_3,h_3,i_3)$, $v : D_2(g_1,g_2,h_2,i_2)$,
and $u : D_2(f,g_1,h_1,i_1)$ a witness for the associativity of the vertical
composition $\assocv(w, v, u)$ in %TODO hbox
\begin{equation*}
\assoc(i_3,i_2,i_1)_*(\assoc(h_3,h_2,h_1)_*(w \circ_1 (v \circ_1 u))) =
(w \circ_1 v) \circ_1 u \text{,}
\end{equation*}
where $\assoc$ is the associativity proof in the 1-skeleton.
The transport is required since the cells at the left and right side of the equation
do not definitionally have the same set of faces.
\item For every $u : D_2(f, g, h, i)$ we need proofs $\idLeftv(u)$ and $\idRightv(u)$
that the following equations hold:
\begin{align*}
\idLeft(i)_*(\idLeft(h)_*(\id_1 \circ_1 u)) &= u \text{ and } \\
\idRight(i)_*(\idRight(h)_*(\id_1 \circ_1 u) &= u \text{.}
\end{align*}
Again, the transports are needed to account for the difference in the faces
in the composition.
\item Of course, we need a horizontal composition and identity:
\begin{align*}
\circ_2 &: \prod_{\ldots} D_2(f,g,h,i) \to D_2(f_2,g_2,i,i_2) 
	\to D_2(f_2 \circ f,g_2\circ,h,i_2) \\
\id_2 &: \prod_{a,b:D_0} ~ \prod_{f : D_1(a,b)} D_2(\id_C(a),\id_C(b),f,f)
\end{align*}

\item Analogously to the ones above, we need the associativiy and identitiy proofs
for the horizontal composition:
\begin{align*}
\assoch &: \assoc(g_3,g_2,g_1)_*(\assoc(f_3,f_2,f_1)_*(w \circ_2 (v \circ_2 u))) =
	(w \circ_2 v) \circ_1 u \text{,} \\ %TODO hbox
\idLefth &: \idLeft(g)_*(\idLeft(f)_*(\id_2 \circ_2 u)) = u \text{, and } \\
\idRighth &: \idRight(g)_*(\idRight(f)_*(\id_2 \circ_2 u) = u \text{,}
\end{align*}
for every $u : D_2(f,g,h,i)$, $v : D_2(f_2,g_2,i,i_2)$, and $w : D_2(f_3,g_3,i_2,i_3)$.
\item For every $f$, $g$, $h$, and $i$, the type of two-cells $D_2(f,g,h,i)$ must
be a set.
\item The identities should distribute over the respective other composition
(compare \ref{eq:linear-ident}):
\begin{align*}
\prod_{a,b,c : D_0} ~ \prod_{f : D_1(a,b)} ~ \prod_{g : D_1(b,c)}
	\id_2 (g \circ f) = \id_2(g) \circ_1 \id_2(f) \\
\prod_{a,b,c : D_0} ~ \prod_{f : D_1(a,b)} ~ \prod_{g : D_1(b,c)}
	\id_1 (g \circ f) = \id_1(g) \circ_2 \id_1(f)
\end{align*}
\item Corresponding to the equation \ref{eq:zero-unique-ident}, we need a proof
that there is only one, unique, zero-square for each point:
\begin{equation*}
\prod_{a : D_0} \id_1(\id_C(a)) = \id_2(\id_C(a))
\end{equation*}
\item Finally, as introduced in equation \ref{eq:interchange}, the interchange
law should hold:
\begin{equation*}
\prod_{\ldots} (x \circ_2 w) \circ_1 (v \circ_2 u) = (x \circ_1 v) \circ_2 (w \circ_1 u)
\end{equation*}
Consider that here, the ``\ldots'' indexing the iterated $\Pi$-type hide a list
9 points in $D_0$, 12 morphisms and four squares!
\end{itemize}
\end{defn}

This list of necessary parameters of a constructor to a double category might seem long at first
glance, but the fact that we implemented the two-cells as dependent types
released us from the duty of adding the cubical identities \ref{eq:corner-ident}
and \ref{eq:degen-ident} propositionally.
Not only those, but also the four first equations in \ref{eq:linear-ident} hold
by definition!
Formulating the definition using three different categories would have rendered
this impossible.
But it is not the case that the notion of the two categories of two-cells are not
accessible in the formalization:

\begin{defn}
We can recover what in definition \ref{def:dbl-cat} we called the \textbf{vertical
precategory} of a double category $D$ as a category $V$ on $A :\equiv
\sum_{(a,b : D_0)} D_1(a,b)$ with morphisms
\begin{equation*}
\hom_V((a,b,f),(c,d,g)) :\equiv \sum_{h:D_1(a,c)} ~ \sum_{i:D_1(b,d)} D_2(f,g,h,i)
\end{equation*}
and composition $(h_2,i_2,v) \circ_V (h_1,i_1,u) :\equiv
(h_2 \circ h_1, i_2 \circ i_1, v \circ_1 u)$.
The corresponding category axioms can easily be proved to follow from the ones
of the 1-skeleton of $D$ and their counterparts in definition \ref{def:dbl-cat-hott}.

The \textbf{horizontal precategory} $H$ of $D$ is defined likewise.
\end{defn}

But how do the basic examples~\ref{def:shell-dbl-cat} of a square double category
and a shell double category translate into HoTT? Stating them is surprisingly
straight-forward:

\begin{example}[Square and shell double category]
The square double category on a category $C$ on a Type $A : \UU$ can be instantiated
as the double precategory having, of course, $C$ as a one skeleton, and setting
$D_2(f,g,h,i) :\equiv \unit$.
By doing this, the type of squares does not contain any more information than its
arguments.
For every quadruple of morphisms forming a square, there is exactly one two-cell.
All necessary conditions to make this a double category hold trivially after
defining $\id_1(f) \equiv \id_2(f) \equiv (u \circ_1 v) \equiv (u \circ_2 v) :\equiv
\star : \unit$.

For the commutative square double category, one might be inclined to set
$D_2(f,g,h,i) :\equiv \squash{g \circ h = i \circ f}$ since each commuting
set of faces should not be inhabited by more than one element.
But since morphisms between given objects form a set, the commutativity witness
is already a mere proposition and we can, without any doubts, define
$D_2(f,g,h,i) :\equiv (g \circ h = i \circ f)$.
Composition and identities can be defined like stated in \ref{def:shell-dbl-cat},
the remaining properties follow easily by the truncation imposed on morphisms
and two-cells.
\end{example}

\begin{defn}(Weak double groupoid)
A \textbf{weak double groupoid} is constructed by giving:
\begin{itemize}
\item A precategory $D$ with objects $D_0$, morphisms $D_1$ and two-cells $D_2$.
\item 
\end{itemize}
\end{defn}

\section{Crossed modules in HoTT}

\section{Presented Types}
% here goes the actual fundamental dbl gpd, xmod

\section{A Seifert-van Kampen theorem for 2-Types}

