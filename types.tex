In this this chapter, I will describe how the structures introduced in the previous
chapter can be translated to homotopy type theory.
Besides formulating the concepts using dependent types this involves caring about
the effects of univalence and proof relevance on these definitions.
What does it mean for two instances of a structure to be propositionally equal?
What truncation levels should be imposed on the parameters of said structure
such that algebraic structures bear no unwanted information in their iterated
equality types?

We start off with defining the notion of a categories, and then continue to
translate the definitions from the previous chapter appropriately.
Finally we will show how to apply the definitions to 2-types and define the
fundamental double groupoid and fundamental crossed module of a presented type.

The standard references for the implementation of categories I will use in this
chapter are the respective chapter of the HoTT-Book~\cite{hottbook} as well as
the paper about ``Univalent Categories and the Rezk Completion'' by Benedikt
Ahrens, Chris Kapulkin and Mike Shulman~\cite{rezk-completion}.
While most of the time I will stick to the (consistent) notation and terminology
of both of these, I will deviate sometimes to bring the presentation more
in line with the actual Lean implementation presented in the next chapter.

\section{Categories in HoTT}
% Here goes stuff about univalent vs non-univalent cats
% maybe rezk-completion

\begin{defn}[Precategory]
Let $A : \UU$ be a type (the \textbf{object type} or \textbf{carrier}). A
\textbf{precategory} $C$ on $A$ is constructed by giving the following data:
\begin{itemize}
\item For each $a, b : A$ a type of morphisms $\hom_C(a, b) : U$ for which we furthermore
require that $\prod_{(a,b:A)} \isSet(\hom_C(a, b))$.
\item The composition of morphisms
\begin{equation*}
\comp_C : \prod_{a,b,c:A} \hom_C(b, c) \to \hom_C(a, b) \to \hom_C(a, c) \text{.}
\end{equation*}
We will most of the leave the first three arguments implicit and just write
$g \circ_C f$ or $gf$ for $\comp_C(a,b,c,g,f)$.
\item An identity operator $\id_C : \prod_{(a:A)} \hom_C(a,a)$.
\item A witness ensuring associativity for all morphisms:
\begin{equation*}
\prod_{a,b,c,d:A} \prod_{h:\hom_C(c,d)} \prod_{g:\hom_C(b,c)} \prod_{f:\hom_C(a,b)}
h \circ_C (g \circ_C f) = (h \circ_C g) \circ_C f
\end{equation*}
\item Witnesses that the identiy morphisms are neutral with respect to composition
from the left and from the right:
\begin{equation*}
\prod_{a, b : A} \prod_{f : \hom_C(a, b)}
\left(\id_C(b) \circ_C f = f\right) \times \left(f \circ_C \id(a) = f\right)
\end{equation*}
\end{itemize}
\end{defn}

As with all the definitions given in this semi-informal style, it is, from a
theoretical standpoint, equivalent whether to see them as a description of an
iterated $\Sigma$-Type or as the only constructor of an inducitve type.
We will later~\ref{rmk:inductives-not-sigmas} see that in formalization practice
it is favorable to choose to introduce them as new inductive types instead of
$\Sigma$-Types.

\section{Double groupoids in HoTT}

\section{Crossed modules in HoTT}

\section{Presented Types}
% here goes the actual fundamental dbl gpd, xmod

\section{A Seifert-van Kampen theorem for 2-Types}

