Making mathematical definitions and theorem proofs readable and verifiable by
computers has become increasingly important in the last years, not only since there
are proofs that are hard or impossible to be checked by a single person due to their
size (one example being Tom Hales' proof of the Kepler conjecture~\cite{flyspeck}).
With the rise of formally verified software, one also wants the same level of trust
for the mathematical theories whose soundness guarantee the correct functionality
of the program.
Fields where formal verification has been successfully used to certify computer
programs include cryptography and aerospace industry.
These rely heavily on results from algebra and calculus and differential equations.

\emph{Homotopy type theory} (HoTT) can serve as a foundation of mathematics
that is better suited to fit the needs of formalizing certain branches of mathematics,
especially the ones of \emph{topology}.
In traditional, set-based approaches to formalizing the world
of mathematical knowledge, topological spaces and their properties
have to be modeled with much effort by referring to the type of real numbers.
In contrast to this, homotopy type theory contains topologically motivated objects
like fibrations and homotopy types as primitives.
This makes it much easier and more natural to reason about topological properties
of these objects.
Homotopy type theory is a relatively new field but it already has produced several
useful implementations and libraries in interactive theorem provers like
Agda~\cite{hott-agda} and Coq~\cite{hott-coq}.
One important feature of homotopy
type theory is that it is \emph{constructive} and thus allows to extract programs
from definitions and proofs.

Homotopy type theory is \emph{proof relevant} which means that there can be distinct
(and internally distinguishable) proofs for one statement.
This leads to the fact that types in HoTT bear the structure of a higher groupoid
in their identities.
The essential problem in the field of \emph{homotopy} is to analyze this structure
of paths and iterated paths between paths in topological spaces or,
in the world of HoTT, in higher types.
This happens by considering the algebraic properties of the homotopy groups or
\emph{homotopy groupoids} of the spaces resp. types.

In his book ``Nonabelian Algebraic Topology''~\cite{nat}, Ronald Brown
introduces the notion of \emph{double groupoids with thin structures} and
\emph{crossed modules over groupoids} to describe the interaction between
the first and the second homotopy groupoid of a space algebraically.
Brown's approach, preceding the discovery of homotopy type theory by a few
decades, is formulated entirely classically and set-based.

In this thesis I describe how I translated some of the central definitions and
lemmas from his book to dependently typed algebraic structures in homotopy type
theory, made them applicable to the analysis of 2-truncated types by creating
the notion of a \emph{fundamental double groupoid of a presented 2-type},
and then formalized them in the newly built interactive theorem proving system
Lean~\cite{lean1}.

The structure of this thesis is as follows: Chapter~\ref{chapter:hott} gives
a short introduction to some basics of homotopy type theory. Chapter~\ref{chapter:nat}
summarizes the considered categories as they are presented in Ronald Brown's book.
Then, Chapter~\ref{chapter:types} describes, how we can translate these
definitions to the setting of homotopy type theory.
Eventually, Chapter~\ref{chapter:lean} tells my experiences in formalizing
the definitions in Lean.

